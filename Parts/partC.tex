\section{Part C: Performance Report/Reflection}

In the realm of live music, where notes hang in the air like suspended dreams, I embarked on two transcendent journeys—one with the legendary Pink Floyd and the other guided by the cinematic genius of Hans Zimmer. These candlelight concerts, each distinct in genre and style, unfolded like ancient scrolls, revealing the magic of sound and light. As the gentle glow of candle flames danced around me, I bore witness to musical alchemy that defied time and space.

\subsection{Pink Floyd Concert: A Progressive Odyssey}
\paragraph{Venue and Date}
On \textbf{January 24th, 2024}, The Great Hall in Toronto, Canada, transformed into a sanctuary for progressive rock enthusiasts. The flickering candlelight cast shadows on the walls, creating an intimate ambiance—a fitting backdrop for the ethereal sounds that awaited us.

\paragraph{Genre and Performers}
Pink Floyd, the architects of cosmic rock, emerged from the shadows. Approximately four musicians—faces obscured by the play of light—stepped into the spotlight. Their instruments whispered secrets of intricate compositions, experimental sounds, and thought-provoking lyrics.

\paragraph{Musical Elements}
The vocal timbre, both delicate and raw, echoed Roger Waters' introspective lyrics. Layers of instrumental texture wove intricate patterns—the bassist anchoring, the keyboardist painting celestial hues, and the lead guitarist conjuring haunting notes. Dynamics shifted seamlessly from hushed whispers to thunderous crescendos, mirroring life's ebb and flow.

\paragraph{Peer-reviewed Reference}
One peer-reviewed reference used is "On Bowie/Shock and Awe: Glam Rock and its Legacy" by Edward Komara\autocite{reynolds}, providing context to the genre of progressive rock.

\paragraph{Audience Reaction}
The audience, visible through the live chat, erupted in applause emojis and exclamations. Their enthusiasm mirrored my own. We journeyed through time, our souls echoing with Pink Floyd's legacy.

\subsection{Hans Zimmer Concert: A Cinematic Overture}
\paragraph{Venue and Date}
A month later, on \textbf{February 26th, 2024}, the Maxwell Meighen Centre in Toronto hosted the Hans Zimmer Candlelight Concert. The warm light bathed the venue, setting the stage for a different kind of magic.

\paragraph{Genre and Performers}
Film music and orchestral compositions unfolded—a genre known for emotive storytelling and dramatic orchestration. Zimmer's scores, etched into our collective memory, evoke powerful emotions during cinematic journeys.

\paragraph{Musical Elements}
The vocal timbre spoke volumes, even without lyrics. The instrumental texture transported us—violin, cello, and percussion painting cinematic landscapes. Dynamics swelled and receded, mirroring our hearts' tides.

\paragraph{Peer-reviewed Reference}
For insights into film music and orchestral compositions, "Marcia J. Citron, Opera on Screen" by Marcia J. Citron\autocite{citron2000opera} provides valuable analysis.

\paragraph{Audience Reaction}
Silence enveloped the room as the final chords of "Time" from \textit{Inception} resonated. The candle flames stood still, as if holding their breath. Zimmer had orchestrated our souls, conducting a symphony of memories and dreams.

\subsection{Summary}
The Pink Floyd and Hans Zimmer candlelight concerts offered profound journeys into the realms of progressive rock and cinematic orchestration, respectively. Each performance, enveloped in the soft glow of candlelight, transcended mere entertainment to evoke a sense of wonder and introspection. Both concerts left indelible impressions, weaving threads of wonder into our souls.

