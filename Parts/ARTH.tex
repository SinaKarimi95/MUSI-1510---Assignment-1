\documentclass[10pt]{beamer}

%% Based on the original theme by Matthias Vogelgesang

\usetheme[progressbar=frametitle]{metropolis}
\usepackage{appendixnumberbeamer}

\usepackage{booktabs}
\usepackage[scale=2]{ccicons}

\usepackage{pgfplots}
\usepgfplotslibrary{dateplot}

\usepackage{xspace}
\newcommand{\themename}{\textbf{\textsc{metropolis}}\xspace}

\title{Street Art in Conflict Zones: Expressions of Resistance and Resilience}
\subtitle{}
\author{Sina Karimi}
\institute{Arth4340}
\date{}

\begin{document}

\maketitle

\begin{frame}{Table of contents}
  \tableofcontents
\end{frame}

\section{Main Topic}
\begin{frame}{Main Topic}
    The main topic of this essay is to explore the role of street art as a form of expression, resistance, and resilience in conflict zones. By examining street art practices in regions affected by conflict, particularly focusing on the case of Gaza (Palestine), the essay will investigate how street art serves as a means for communities to assert their identities, challenge oppressive regimes, and cope with trauma amidst adversity.
\end{frame}

\section{Research Questions}
\begin{frame}{Research Questions}
\begin{itemize}
    \item How does street art function as a form of cultural resistance and expression in conflict-affected areas?
    \item What socio-political implications does street art have in shaping collective memory and fostering community solidarity in conflict zones?
    \item How does street art contribute to the resilience and empowerment of marginalized communities living amidst conflict?
    \item What challenges do street artists face in creating and showcasing their work in conflict zones, particularly in the case of Gaza (Palestine)?
\end{itemize}
\end{frame}

% Additional frames and sections can be added following the same structure

\end{document}
